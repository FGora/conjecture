\documentclass[11pt,a4paper]{amsart}  

\usepackage[a4paper]{geometry}
\geometry{verbose,tmargin=3.0 cm,bmargin=3.0 cm,lmargin=2.5 cm,rmargin=2.5 cm}


\usepackage[utf8]{inputenc}
\usepackage[english]{babel}


\usepackage{amsthm}
\usepackage{amstext}
\usepackage{amsmath}
\usepackage{amsfonts}
\usepackage{amssymb}
\usepackage{mathtools}


\usepackage{amsmath, amssymb, amsfonts}
\usepackage[arrow, matrix, curve]{xy}
\usepackage{enumitem} 
\usepackage{color}
\usepackage[hypertexnames=false]{hyperref}
\usepackage[utf8]{inputenc}
\usepackage{todo}
\usepackage{mathdots}
\usepackage{appendix}
\usepackage{pdflscape}


\usepackage{MnSymbol}

\renewcommand{\labelenumi}{(\roman{enumi})}







\newtheorem{prop}{Proposition}[section]
\newtheorem{conj}{Conjecture}[section]
\newtheorem{lemma}[prop]{Lemma}
\newtheorem{corollary}[prop]{Corollary}
\newtheorem{constr}[prop]{Construction}
\newtheorem{thm}[prop]{Theorem}
\newtheorem{example}[prop]{Example}

\newtheorem*{thm*}{Theorem}
\newtheorem*{prop*}{Proposition}
\newtheorem*{definition*}{Definition}

\newtheorem*{acknowledge}{Acknowledgements}

\newtheorem{innercustomthm}{Theorem}
\newenvironment{customthm}[1]
  {\renewcommand\theinnercustomthm{#1}\innercustomthm}
  {\endinnercustomthm}
\newtheorem{innercustompro}{Proposition}
\newenvironment{custompro}[1]
  {\renewcommand\theinnercustompro{#1}\innercustompro}
  {\endinnercustompro}

\theoremstyle{definition}
\newtheorem{definition}[prop]{Definition}
\newtheorem*{ex}{Example}

\theoremstyle{remark}
\newtheorem{rem}[prop]{Remark}




\newcommand{\Z}{\mathbb{Z}}
\newcommand{\N}{\mathbb{N}}
\newcommand{\R}{\mathbb{R}}
\newcommand{\Q}{\mathbb{Q}}
\newcommand{\F}{\mathbb{F}}
\newcommand{\C}{\mathbb{C}}
\newcommand{\A}{\mathbb{A}}
\newcommand{\Gm}{\mathbb{G}_m}
\newcommand{\Projsp}{\mathbb{P}}
\newcommand{\alc}{\mathrm{Alcov}}
\newcommand{\alcn}{\mathrm{Alcov}_n}

\newcommand{\rg}{\mathcal{O}}
\newcommand{\res}{\kappa}
\newcommand{\cres}{\overline{\kappa}}
\newcommand{\unif}{\pi}
\newcommand{\quot}{\mathrm{K}}

\newcommand{\im}[1]{\mathrm{Im}(#1)}
\newcommand{\iml}[2]{\mathrm{Im}_{#1}\left(#2\right)}

\newcommand{\id}[1]{\mathrm{Id}_{ #1 }}
\newcommand{\gln}{\mathrm{GL}_n}

\newcommand{\affw}{W^{\mathrm{aff}}}
\newcommand{\grz}{\mathrm{Gr}_{n,r}(\res)}
\newcommand{\grs}{\mathrm{Gr}^{n,r}}
\newcommand{\grg}{\mathrm{Gr}_{n,r}(\quot)}
\newcommand{\grb}[1]{\widetilde{\mathrm{Gr}_{n,r}(\rg)}_{\leq #1}}
\newcommand{\grzs}[1]{\mathrm{Gr}^{#1}(\res)}
\newcommand{\grlat}[2]{\mathrm{Gr}^{#1}\left(#2\right)}
\newcommand{\gr}[1]{\mathrm{Gr}^k\left({#1}\right)}

\newcommand{\can}{\mathcal{M}}
\newcommand{\cangoe}{\widetilde{\mathcal{G}}}
\newcommand{\cangen}{\mathcal{G}}
\newcommand{\cangenpl}{\cangen^{\mathrm{pl}}}
\newcommand{\cangenstr}{\mathcal{S}}
\newcommand{\cangenstrpl}{\mathcal{S}^{\mathrm{pl}}}

\newcommand{\adm}{\mathrm{Adm}(\mu)}
\newcommand{\set}[1]{[#1]}
\newcommand{\sset}{\binom{[n]}{k}}

\newcommand{\defeq}{\mathrel{\mathop:}=}
\newcommand{\loc}{\mathcal{M}^{\text{loc}}}
\newcommand{\glsec}[1]{\mathcal{O}_{#1}}

\newcommand{\spec}[1]{\mathrm{Spec}\left(#1\right)}
\newcommand{\bl}[2]{\mathrm{Bl}_{#1}\left(#2\right)}
\newcommand{\proj}[1]{\mathrm{Proj}\left(#1\right)}
\newcommand{\Bl}[2]{\mathrm{BL}_{#1}\left(#2\right)}
\newcommand{\ini}[1]{\left(#1\right)^{\mathrm{in}}}
\newcommand{\val}[1]{\mathfrak{v}\left(#1\right)}
\newcommand{\degre}[1]{\mathrm{deg}\left(#1\right)}
\newcommand{\tot}[1]{#1^{\mathrm{tot}}}
\newcommand{\str}[1]{#1^{\mathrm{s}}}
\newcommand{\mon}[1]{#1^{\mathrm{mon}}}
\newcommand{\schu}[1]{X_{\{#1\}}}
\newcommand{\sing}[1]{#1^{\mathrm{sing}}}
\newcommand{\schuvar}[1]{X_{#1}}
\newcommand{\schucell}[1]{\schuvar{#1}^\circ}

\newcommand{\M}[1]{\mathcal{M}_{\Projsp}\left(#1\right)}
\newcommand{\Mgr}[2]{\mathcal{M}_{\mathrm{Gr}^#1}\left(#2\right)}
\newcommand{\aut}[1]{\mathrm{Aut}\left(#1\right)}
\newcommand{\chn}{\mathcal{L}}

\newcommand{\projectbar}[1]{\mathrm{\overline{\pr_{\Projsp}^{#1}}}}
\newcommand{\projbar}{\mathrm{\overline{\pr_{\Projsp}}}}
\newcommand{\project}[1]{\mathrm{\pr_{\Projsp}^{#1}}}
\newcommand{\pr}{\mathrm{pr}}

\newcommand{\projectbargr}[1]{\mathrm{\overline{\pr_{\mathrm{Gr}}^{#1}}}}
\newcommand{\projbargr}{\mathrm{\overline{\pr_{\mathrm{Gr}}}}}
\newcommand{\projectgr}[1]{\mathrm{\pr_{\mathrm{Gr}}^{#1}}}
\newcommand{\muspro}[2]{\pr_{#1 , #2}}
\newcommand{\musprol}[1]{\pr_{#1}}

\newcommand{\basealcov}{\mathrm{a}_b}

\newcommand\lt{[\hspace{-.12em}[ t ]\hspace{-.12em} ] } 
\newcommand\lT{(\!( t )\!)}

\title{A conjecture on irreducible components of certain Mustafin varieties}


\begin{document}
\maketitle
It is of great interest to to specify an explicit bijection on the sets of irreducible components of $\M{\overline{\bigwedge^k\Gamma^{\mathrm{st}}}}_{\res}$ and $\M{{\bigwedge^k\Gamma^{\mathrm{st}}}}_{\res}$. This is done in the conjecture below.  
  

\begin{conj}
\label{conj_convexcl_bijection}
	We get a bijection
	\begin{align*}
		\left\{C\middle\vert C \text{ irr. component of } \M{\overline{\bigwedge^k\Gamma^{\mathrm{st}}}}_{\res} \right\}&\longrightarrow \left\{C\middle\vert C \text{ irr. component of } \M{\bigwedge^k\Gamma^{\mathrm{st}}}_{\res} \right\}\\
		C & \longmapsto \projbar (C) 
	\end{align*}
	where $\projbar\colon \M{\overline{\bigwedge^k\Gamma^{\mathrm{st}}}}\rightarrow \M{\bigwedge^k\Gamma^{\mathrm{st}}}$ is the natural projection.
\end{conj}

\begin{rem}
	It is well known that for an irreducible component $C$ of $\M{\bigwedge^k\Gamma^{\mathrm{st}}}_{\res}$ there exist a unique irreducible component $\overline{C}$ of $\M{\overline{\bigwedge^k\Gamma^{\mathrm{st}}}}_{\res}$ surjecting to $C$. Hence to prove the conjecture we just need to show that the images $\projbar (C)$ are irreducible components.
\end{rem}

As evidence for the conjecture we can calculate some cases of low dimensions and prove the following proposition. 

\begin{prop}
\label{prop_conj_k_2}
	For $k=2$ Conjecture \ref{conj_convexcl_bijection} is true.
\end{prop}

The proof of this proposition will occupy us for the rest of this section therefore let us first indicate one important immediate consequence of the Conjecture. Furthermore we will describe a method to approach the conjecture in general before we prove the proposition.

\begin{thm}
   \label{thm_components_linear_subsp}
	Assume Conjecture \ref{conj_convexcl_bijection}. Then we get a bijection
	\begin{align*}
		\left\{C\middle\vert C \text{ irr. component of } \M{\bigwedge^k\Gamma^{\mathrm{st}}}_{\res} \right\}&\longrightarrow \left\{\Projsp (V_I)\subseteq \Projsp(\bigwedge^k \Lambda_0)_{\res}\middle\vert I \in \binom{[n]}{k} \right\}\\
		C & \longmapsto \project{0} (C)
	\end{align*}
	where $\project{0}\colon \M{\bigwedge^k\Gamma^{\mathrm{st}}}\rightarrow \Projsp(\bigwedge^k \Lambda_0)$ is the natural projection. 
	And moreover for a linear subspace $\Projsp(V_I)$ for $I\in \binom{[n]}{k}$ the inverse image $(\project{0})^{-1}(\Projsp(V_I) )$ is the union $\bigcup_{J\leq I} C_J$ of irreducible components. 
\end{thm}


\begin{rem}
	In \cite{AL17} a combinatorial method was described to compute dimensions of certain images of rational maps using a result by \cite{Li18}. To describe this method and the implication we want to use, we need the following setup. Note that we are using the dual notion of projective space compared to the reference.\\
	Fix a finite set of lattice classes $\Gamma$ in $\quot^n$, an irreducible component $C$ of $\M{\Gamma}_{\res}$ and a class $[\Lambda]$ in $\Gamma$. Take a representative $\Lambda=\langle \unif^{m_I (\Lambda )} e_I \rangle$ of $[\Lambda]$ and a representative $\Lambda_C= \langle \unif^{m_I(\Lambda_C )}e_I\rangle$ of the class corresponding to the irreducible component $C$. Choose $\Lambda_C$ to be maximal with $\Lambda_C \subseteq\Lambda$. We define the subset 
	\begin{align*}
		W_{\Lambda}\defeq\{i\in[n]\vert m(\Lambda)_i-m(\Lambda_C )_i< \max\limits_{j\in [n]}\{m(\Lambda)_j-m(\Lambda_C )_j\} \}
	\end{align*}  of $[n]$ and construct the set 
	\begin{align*}
		M(h,C)\defeq\{(a_\Lambda)_{[\Lambda]\in \Gamma}\in\N^{\Gamma}\vert \sum_{\Gamma}a_\Lambda =h\text{ and } n-\sum_{\Lambda\in  I}a_\Lambda > \sharp\bigcap_{\Lambda\in I}W_{\Lambda}\text{ for all subsets } \emptyset\neq I \subseteq\Gamma \}.
	\end{align*} 
	In the following proposition we describe how this combinatorial data encodes information of the image $\projbar (C)$.
\end{rem}

\begin{prop}(\cite{Li18} and  \cite[Theorem 2.18]{AL17})
\label{prop_dimension_computed_with_subsets}
	For a finite set of lattice classes $\Gamma$ and an irreducible component $C$ of $\M{\overline{\Gamma}}_{\res}$ the dimension of $\projbar (C)\subseteq \M{{\Gamma}}_{\res}$ is computed by
	\begin{align*}
		\dim(\projbar (C))= \max\{h\vert M(h,C)\neq \emptyset \}.
	\end{align*} 
\end{prop}

\begin{rem}
\label{rem_idea_proof_conjecture}
	Let us take the time to explain an idea to approach Conjecture \ref{conj_convexcl_bijection} using the proposition above. Fix an irreducible component $C$ of $\M{\overline{\bigwedge^k\Gamma^{\mathrm{st}}}}$ and take the corresponding class $[\Lambda_C]$ in $\overline{\bigwedge^k\Gamma^{\mathrm{st}}}$. 
%	First we recall that using Lemma \ref{lem_irr_comp_birational_map} the image of $C$ in $\M{\overline{\bigwedge^k\Gamma^{\mathrm{st}}}}_{\res}$ is clearly an irreducible component if $[\Lambda_C]$ is already in $\bigwedge^k\Gamma$. Hence we just have to check the cases where $\Lambda$ is in $\overline{\bigwedge^k\Gamma^{\mathrm{st}}}\setminus {\bigwedge^k\Gamma^{\mathrm{st}}}$.
	 \\
%	Now we can find a subset $\Gamma_C$ of $\bigwedge^k\Gamma^{\mathrm{st}}$ minimal such that $\overline{\Gamma_C}$ contains $[\Lambda_C]$. With out loss of generality we can further assume that the homothety class of $\bigwedge^k\Lambda_0$ is contained in $\Gamma_C$. By Lemma \ref{lem_irr_comp_birational_map} and Lemma \ref{lem_convex_irr_com_correspond_to_lattices} the image of $C$ in $\M{\Gamma_C}_{\res}$ is an irreducible component. And using Lemma \ref{lem_irr_comp_birational_map} again we see that the image of $C$ in $\M{\Gamma_C}_{\res}$ is an irreducible component if and only if the image in $\M{\bigwedge^k\Gamma^{\mathrm{st}}}_{\res}$ is an irreducible component. \\
	But since the image of $C$ in $\M{\Gamma_C}$ is irreducible, it is an irreducible component if its dimension is $\binom{[n]}{k}-1$. Using Proposition \ref{prop_dimension_computed_with_subsets} this is equivalent to $M(\binom{[n]}{k}-1,C)$ not being empty for the set $\Gamma_C$ of lattice classes.\\
	In general the sets $\Gamma_C$ can be difficult to determine, but for $k=2$ we have the following easy description. 
	\end{rem}


\begin{lemma}
\label{lem_convex_closure_k_2}
	For $[\Lambda]\in \overline{\bigwedge^2 \Gamma^{\mathrm{st}}}$ there are classes $[\Lambda_1]$ and $[\Lambda_2]$ in $\bigwedge^2 \Gamma^{\mathrm{st}}$ such that $[\Lambda]$ is in the convex closure $\overline{\left\{[\Lambda_1],[\Lambda_2]\right\}}$.
\end{lemma}

\begin{proof}
	Take $[\Lambda]$ in $\overline{\bigwedge^2 \Gamma^{\mathrm{st}}}$ arbitrary. Then $\Lambda$ is of the form $\bigcap_{i\in I} \unif^{n_i}\bigwedge^2\Lambda_i$ for some $I\subseteq[n]$ and without loss of generality we have $0\in I$, $n_0=0$ and $n_i> 0$ for all $i\in I\setminus\{0\}$. Further assume that $I$ is minimal, i.e. $\Lambda$ is properly contained in $\bigcap_{i\in J} \unif^{n_i}\bigwedge^2\Lambda_i$ for all proper subsets $J\subset I$. Now we note that for all $i\in I$ we have $\unif^{n_i}\bigwedge^2\Lambda_i\subseteq \bigwedge^2\Lambda_0$ if $n_i\geq 2$. Using minimality of $I$ we conclude that $n_i=1$ for all $i\in I\setminus \{0\}$. But since $p \bigwedge^2\Lambda_i\subseteq \unif\bigwedge^2\Lambda_j$ for $i\leq j$ we again see by minimality of $I$ that $\sharp I\leq 2$.
\end{proof}


\begin{proof}[Proof of Proposition \ref{prop_conj_k_2} ]
	Fix an irreducible component $C$ of $\M{\bigwedge^2\Gamma^{\mathrm{st}}}$. Following the idea described in Remark \ref{rem_idea_proof_conjecture} and Lemma \ref{lem_convex_closure_k_2} we just have to prove $M(\binom{n}{2}-1,C)\neq \emptyset$ for\linebreak $\Gamma=\left\{[\bigwedge^2 \Lambda_0],[\bigwedge^2\Lambda_i]\right\}$ for some $i\in [n]$.\\
	But for $\unif\bigwedge^2 \Lambda_0\cap\bigwedge^2\Lambda_i=\langle \unif^{m^{i}_I}e_I\vert m^{i}_I= \max\{\sharp(I\cap [i]),1\} \rangle_{I\in \binom{[n]}{k} }$ hence $W_{\bigwedge^2\Lambda_0}= \left\{I\vert m^i_{I}>1 \right\}$ and $W_{\bigwedge^2\Lambda_i}= \left\{I\vert m^i_{I}<1 \right\}$. Now for $a_{\bigwedge^2 \Lambda_0}\defeq \sharp W_{\bigwedge^2 \Lambda_i}$ and $a_{\bigwedge^2 \Lambda_i}\defeq \sharp W_{\bigwedge^2 \Lambda_0}+ \sharp\left\{I\vert m^i_{I}=1 \right\}-1$ we have an element $(a_{\bigwedge^2 \Lambda_0},a_{\bigwedge^2 \Lambda_i})\in M(\binom{n}{2}-1,C)$.

\end{proof}

\bibliographystyle{alpha}
\bibliography{literatur}

\end{document}
